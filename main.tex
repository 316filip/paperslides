% !TeX root = main.tex

\input style
\slides
%\wide
\cslang

\tit Prezentace
\subtit Filip Rund
\subtit \today
\pg;

\chap O čem je tato prezentace?
\chapcnt
\pg;

\sec Téma této prezentace
* Tato prezentace nemá žádné téma
* Jejím cílem je předvést šablonu designu
* I tento text je pouze výplňový
\begitems
* na prázdném snímku se totiž design špatně ukazuje
\enditems
* Šablonu lze snadno použít zahrnutím řádku \code{\\input style} na začátek \OpTeX\ souboru
* Použití šablony usnadní tvorbu přehledné a pohledné prezentace
\pg;

\chap Název oddílu
\small{Oddíly jsou číslovány vpravo dole}
\chapcnt
\pg;

\sec Výhody a nevýhody rozložení do dvou sloupců
\cols{
    {\bf Výhody:}
    * Lepší přehlednost
    * Ideální pro seznamy výhod a nevýhod
    * Může být využit celý prostor na snímku
}{
    {\bf Nevýhody:}
    * Na jeden řádek se vejde méně textu
    * Mám číst dříve sloupce, nebo řádky?
    * Může vést k zahlcení informacemi
}
\pg;

\chap Obrázky
\chapcnt
\pg;

\picquote{quote.png}{180pt} Toto místo je jako dělané pro nějaký dojemný motivační citát
\pg;

\sec Další slide s obrázkem
\centerpic{img.png}{250pt}
\center{Pokud obrázek potřebuje bližší vysvětlení, je tu od toho tento popis}
\pg;

\chap Další maličkosti
\chapcnt
\pg;

\shout Hejhola!
\center{Tento výkřik může sloužit k upoutání pozornosti na nějakou důležitou informaci.}
\pg;

\sec Zvýraznění
Důležitý text lze \highlight{zvýraznit} příkazem \code{\\highlight{}}.
\medskip
Některá slova lze také \emph{zdůraznit} příkazem \code{\\emph{}}.
\medskip
Text lze také \small{zmenšit} uvnitř \code{\\small{}}.
\medskip
\center{\code{\\center{}} dovede zarovnat předaný obsah na střed snímku.}
\pg;

\chap Blíží se konec
\small{Počítadlo oddílů vpravo dole právě uzavřelo prvních 5}
\chapcnt
\pg;

\sec Seznam vytvořených maker
* \code{\\center{text}} – zarovnání na střed
* \code{\\chapcnt} – výpis počtu oddílů (čárkami)
* \code{\\picquote{filename}{width} quote text} – zobrazení obrázku a citátu vedle sebe
* \code{\\centerpic{filename}{width}} – vložení obrázku na střed snímku
* \code{\\cols{left column}{right column}} – rozdělení obsahu do dvou polovin snímku
* \code{\\shout text} – obří text na středu snímku
* \code{\\highlight{text}} – žlutý podkres textu
* \code{\\emph{text}} – žluté zdůraznění textu
* \code{\\small{text}} – zmenšení textu
\pg;

\chap Závěr
\chapcnt
\pg;

\sec Děkuji za pozornost!
\centerline{Nějaké otázky?}
\medskip
\centerline{rundfili@fit.cvut.cz}
\pg;

\sec Zdroje
    {\bf Vědomosti:}
* \url{https://petr.olsak.net/optex/}
* \url{https://www.luatex.org/}
\medskip
{\bf Obrázky:}
* \url{https://pixabay.com/cs/photos/pap\%C3\%ADr-textura-balic\%C3\%AD-pap\%C3\%ADr-pozad\%C3\%AD-3155438/}
* \url{https://pixabay.com/cs/illustrations/planeta-sci-fi-sv\%C4\%9Bt-fantazie-sen-1702788/}
* \url{https://pixabay.com/cs/illustrations/mo\%C5\%99e-slunce-mraky-krajina-malov\%C3\%A1n\%C3\%AD-7456253/}
\pg.